\documentclass{techreport}

% Common packages
\usepackage[utf8]{inputenc}
\usepackage{amsmath,amssymb,amsfonts}
\usepackage{algorithm}
\usepackage{algorithmic}
\usepackage{enumitem}

% Custom commands
\newcommand{\eg}{e.g.}
\newcommand{\ie}{i.e.}
\newcommand{\etc}{etc.}
\newcommand{\etal}{et~al.}

\title{Your Technical Report Title}

\author{Author One}
\author{Author Two}
\author{Author Three}

\affiliation{LMMs Lab}
\affiliation{Institution Name}

\abstract{%
This is where you write your abstract. The abstract should provide a concise summary of your work, including the problem statement, methodology, key results, and conclusions. Keep it to approximately 150-250 words.
}

\metadata[Code]{\url{https://github.com/your-repo}}
\metadata[Data]{\url{https://huggingface.co/your-data}}
\metadata[Model]{\url{https://huggingface.co/your-model}}

\begin{document}

\maketitle

\section{Introduction}
\label{sec:intro}

Your introduction goes here. Describe the problem you are addressing, why it is important, and provide an overview of your approach.

\section{Related Work}
\label{sec:related}

Discuss prior work relevant to your research. Cite relevant papers using \texttt{\\citep\{key\}} for parenthetical citations or \texttt{\\citet\{key\}} for textual citations.

\section{Method}
\label{sec:method}

Describe your methodology in detail. Include figures and equations as needed.

\begin{equation}
    \mathcal{L} = \sum_{i=1}^{N} \ell(f(x_i), y_i)
\end{equation}

\subsection{Architecture}

Describe your model architecture or system design.

\subsection{Training}

Describe your training procedure, hyperparameters, and optimization strategy.

\section{Experiments}
\label{sec:experiments}

\subsection{Experimental Setup}

Describe your experimental setup, datasets, baselines, and evaluation metrics.

\subsection{Main Results}

Present your main results. Use tables and figures to clearly communicate your findings.

\begin{table}[h]
\centering
\caption{Main results on benchmark datasets.}
\label{tab:main}
\begin{tabular}{lcc}
\toprule
Method & Dataset A & Dataset B \\
\midrule
Baseline 1 & 75.2 & 82.1 \\
Baseline 2 & 78.4 & 84.3 \\
\textbf{Ours} & \textbf{82.1} & \textbf{87.5} \\
\bottomrule
\end{tabular}
\end{table}

\subsection{Ablation Studies}

Present ablation studies to analyze the contribution of different components.

\section{Conclusion}
\label{sec:conclusion}

Summarize your contributions and discuss limitations and future work.

\section*{Acknowledgments}

Acknowledge funding sources and collaborators.

\bibliographystyle{plainnat}
\bibliography{references}

\appendix
\section{Additional Results}
\label{sec:appendix}

Include additional experimental results, proofs, or implementation details.

\end{document}
